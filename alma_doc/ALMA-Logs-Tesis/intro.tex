\section{Acerca de  ALMA}
El proyecto ALMA (Atacama Large Millimeter/submillimeter Array) es un trabajo colaborativo entre el European Southern Observatory (ESO), el National Radio Astronomy Observatory (NRAO) y el National Astronomical Observatory of Japan (NAOJ). Este gran proyecto astronómico consiste en un arreglo de 66 antenas ubicado en el desierto de Atacama, en el norte de Chile.
ALMA tiene como propósito observar el cielo, en búsqueda de nuestros orígenes cósmicos. Esta observación es guiada por científicos y astrónomos del mundo, quienes solicitan espacios de observación.
ALMA Common Software (ACS) provee una plataforma para todos los sistemas del observatorio, la cual se basa en un modelo de componentes de software distribuido. Asimismo, ACS genera un archivo de registros (log) con eventos que dan cuenta del funcionamiento del sistema, tales como: errores, alertas, información de ejecución de programas, interrupciones del sistema, etc.\citep{ESO}.

\subsection{Problemática General}
ACS es un software distribuido altamente complejo. Como toda infraestructura de software, este presenta fallas en su funcionamiento las que pueden llegar a provocar la detención completa del sistema, provocando retrasos en las tareas de observación.
Gran parte del funcionamiento de ACS es permanentemente registrado en archivos de logs, los cuales suman diariamente grandes cantidades de información sobre la operación y eventos asociados a diferentes niveles de alerta. Sin embargo, dada la gran cantidad de registros generados por este sistema, resulta complejo el poder analizar estos datos en forma eficiente y efectiva.
 
Alma tiene un infraestructura altamente compleja, lo que se refleja en sus sistema de generación de logs, por lo que es natural encontrar problemáticas asociadas a numerosas causas y por este motivo es necesario buscar y analizar estas posibles problemáticas. El problema se plantea complejo tanto por la diversidad de los logs como por su cantidad.
 
Por otra parte, se ha desarrollado herramientas y metodologías de minería de datos que tienen por objetivo el apoyar el análisis de grandes volúmenes de datos de manera de poder extraer conocimientos nuevos y útiles. Estas herramientas son capaces de procesar grandes volúmenes de información y encontrar relaciones entre las múltiples variables que afectan los procesos. Para poder aplicar estas herramientas es necesario plantear objetivos de búsqueda y aplicar una metodología de descubrimiento de conocimiento en bases de datos (Knowledge Discovery in Databases, KDD).
 
Sin embargo, debido a la alta complejidad del sistema ACS, los objetivos de búsqueda que guían el análisis son múltiples, lo que requiere de una comprensión muy acabada de todo el proceso para poder modelar algunas partes a través de la minería de datos y buscar entonces las relaciones que permiten comprender por ejemplo, porqué y bajo qué circunstancias el sistema falla. El estudio y comprensión del sistema ACS permitirá la identificación de un problema específico para el cual es posible plantear una aproximación basada en minería de datos.



\newpage

 
\subsection{Objetivo General}

\begin{itemize}
	\item Identificar una problemática y generar una propuesta para aplicar una metodología de minería de datos aplicado a los registro de logs de ACS, que de solución al problema identificado.
\end{itemize}


\subsection{Objetivos Específicos}
\begin{itemize}	
	\item Identificar un problema relevante para ALMA y que pueda ser abordado analizando los logs del sistema ACS.
	\item Establecer los requisitos, objetivos y alcances de la problemática a analizar.
	\item Plantear una metodología, basada en el proceso KDD, para abordar la problemática identificada.
	\item Formular un proyecto para resolver el problema planteado.
\end{itemize}



